%!TEX program = pdflatex
% Full chain: pdflatex -> bibtex -> pdflatex -> pdflatex
\documentclass[lang=en,12pt]{elegantpaper}
\usepackage{url}
\usepackage[binary-units=true]{siunitx}
\newcommand{\mysec}[1] {
  \SI[per-mode=symbol]{#1}{\second}
}


\title{Gomoku AI Based On Deep Learning}
\author{Yechang WU (11711918), You Lin (11711809)}
\institute{Southern University of Science and Technology}
\date{\today}

\begin{document}

\maketitle

\begin{abstract}
  Gomoku, also called Five in a Row, is an abstract strategy board game.
  It is traditionally played with Go pieces (black and white stones) on a Go board, using 15$\times$15 of the 19$\times$19 grid intersections.
  The game is known in several countries under different names.While playing Gomoku, players alternate turns placing a stone of their color on an empty intersection. The winner is the first player to form an unbroken chain of five stones horizontally, vertically, or diagonally.

  \keywords{Gomoku, Game AI, Deep neural network, Reinforcement learning}
\end{abstract}

\section{Introduction}
\subsection{topic selected}
a)	 : the application of the Gomoku based on the tensorflow.
\subsection{data selection}
\subsection{model selection}
\subsection{performance goal}
a)	can win AI on the internet.
\subsection{Prerequisites}
\subsection{training process}
a)	deploy the program of AI written in tensorflow into the ModelArt,
b)	test the performance of the model with the times go
\subsection{the method of evaluation}
a)	let the AI play with another AI on the internet and take the win rate of a large number of competition into consideration

% \nocite{*}
% \bibliography{wpref}

\end{document}
